\begin{titlepage}
    \begin{center}
        %{\large \MakeUppercase{Pontifical Catholic University of São Paulo}}\\
        {\large \MakeUppercase{Pontifícia Universidade Católica de São Paulo}}\\
        {\large \MakeUppercase{PUC-SP}}\\
        \vspace{5cm}
        {\large \MakeUppercase{Rogério Yamada}}\\
        \vspace{5cm}
        %{\Large \MakeUppercase{AI-Powered Tools as Academic English Writing Assistants to English-as-L2 Authors: A Multi-Dimensional Analysis}}\\
        {\Large \MakeUppercase{Ferramentas Baseadas em Inteligência Artificial como Assistentes de Escrita Acadêmica em Inglês para Autores em Inglês como Segunda Língua: Uma Análise Multi-Dimensional}}\\
        \vspace{5cm}
        %{\large \MakeUppercase{Master of Arts in Applied Linguistics and Language Studies}}
        {\large \MakeUppercase{Mestrado em Linguística Aplicada e Estudos da Linguagem}}
        \vfill
        {\large \MakeUppercase{São Paulo}}
        
        {\large 2025}
    \end{center}
\end{titlepage}

\begin{titlepage}
    \begin{center}
        %{\large \MakeUppercase{Pontifical Catholic University of São Paulo}}\\
        {\large \MakeUppercase{Pontifícia Universidade Católica de São Paulo}}\\
        {\large \MakeUppercase{PUC-SP}}\\
        \vspace{5cm}
        {\large \MakeUppercase{Rogério Yamada}}\\
        \vspace{5cm}
        %{\Large \MakeUppercase{AI-Powered Tools as Academic English Writing Assistants to English-as-L2 Authors: A Multi-Dimensional Analysis}}\\
        {\Large \MakeUppercase{Ferramentas Baseadas em Inteligência Artificial como Assistentes de Escrita Acadêmica em Inglês para Autores em Inglês como Segunda Língua: Uma Análise Multi-Dimensional}}\\
        \vspace{3cm}
        \begin{adjustwidth}{0.5\textwidth}{0cm}
            %Dissertation submitted to the dissertation committee of the Pontifical Catholic University of São Paulo in partial fulfillment of the requirements for the degree of Master of Arts in Applied Linguistics and Language Studies, under the supervision of Professor Tony Berber Sardinha, Ph.D.
            Dissertação apresentada à banca examinadora da Pontifícia Universidade Católica de São Paulo como exigência parcial para obtenção do título de Mestre em Linguística Aplicada e Estudos da Linguagem, sob orientação do Prof. Dr. Antonio Paulo Berber Sardinha.
        \end{adjustwidth}
        \vfill
        {\large \MakeUppercase{São Paulo}}
        
        {\large 2025}
    \end{center}
\end{titlepage}

\begin{titlepage}
    %I authorise, exclusively for academic and scientific purposes, the total or partial reproduction of this master's dissertation by photocopying or electronic processes.
    Autorizo, exclusivamente para fins acadêmicos e científicos, a reprodução total ou parcial desta dissertação de mestrado por processos fotocopiadores ou eletrônicos.
    
    \vspace{2cm}
    %Signature: \makebox[0.75\textwidth]{\hrulefill}\\
    Assinatura: \makebox[0.75\textwidth]{\hrulefill}\\
    %Date: [work in progress]\\
    Data: [work in progress]\\
    E-mail: eyamrog@gmail.com\\
    %Lattes Curriculum: \href{http://lattes.cnpq.br/9204618373258699}{http://lattes.cnpq.br/9204618373258699}\\
    Curriculo Lattes: \href{http://lattes.cnpq.br/9204618373258699}{http://lattes.cnpq.br/9204618373258699}\\
    \vfill
    \begin{center}
        % Gerenciador de ficha catalográfica
        % http://biblio2.pucsp.br/ficha/?_ga=2.154384056.1415767632.1628681585-1429258994.1628681585
        \begin{tabular}{|p{0.85\linewidth}|}
            \hline
            %[Bibliographic information -- to be included after dissertation defence]\\
            [Informação bibliográfica -- a ser incluída após a defesa da dissertação]\\
            \\ \hline
        \end{tabular}
        %\begin{tabular}{|p{0.08\linewidth} p{0.77\linewidth}|}
        %    \hline
        %     & \texttt{\small Yamada, Rogério} \\
        %    \texttt{\small Y19a} & \texttt{\small AI-Powered Tools as Academic English Writing Assistants to English-as-L2 Authors: A Multi-Dimensional Analysis. /} \\
        %     & \texttt{\small Rogério Yamada. --- São Paulo: [s.n.], 2025.} \\
        %     & \texttt{\small 120p. il.} \\
        %     & \\
        %     & \texttt{\small Supervisor: Professor Tony Berber Sardinha, Ph.D..} \\
        %     & \texttt{\small Dissertation (Master's) --- Pontifical Catholic University of São Paulo, Applied Linguistics and Language Studies Graduate Programme.} \\
        %     & \\
        %     & \texttt{\small 1. AI-generated language. 2. English academic register. 3. English-as-L2 authors. 4. Multi-Dimensional Analysis. I. Berber Sardinha, Ph.D., Professor Tony. II. Pontifical Catholic University of São Paulo, Applied Linguistics and Language Studies Graduate Programme. III. Title.} \\
        %     & \\
        %     & \hspace{8cm} \texttt{\small CDD 410} \\
        %     \hline
        %\end{tabular}
    \end{center}
\end{titlepage}

\begin{titlepage}
    Rogério Yamada\\
    %AI-Powered Tools as Academic English Writing Assistants to English-as-L2 Authors: A Multi-Dimensional Analysis\\
    Ferramentas Baseadas em Inteligência Artificial como Assistentes de Escrita Acadêmica em Inglês para Autores em Inglês como Segunda Língua: Uma Análise Multi-Dimensional\\
    %Approved: [work in progress]
    Aprovada em: [work in progress]

    \vspace{2cm}
    %Dissertation submitted to the dissertation committee of the Pontifical Catholic University of São Paulo in partial fulfillment of the requirements for the degree of Master of Arts in Applied Linguistics and Language Studies, under the supervision of Professor Tony Berber Sardinha, Ph.D.
    Dissertação apresentada à banca examinadora da Pontifícia Universidade Católica de São Paulo como exigência parcial para obtenção do título de Mestre em Linguística Aplicada e Estudos da Linguagem, sob orientação do Prof. Dr. Antonio Paulo Berber Sardinha.

    \vspace{4cm}
    %Dissertation Committee
    Banca Examinadora

    \vspace{3cm}
    \makebox[0.75\textwidth]{\hrulefill}\\
    %Prof. Marilisa Shimazumi, Ph.D.
    Prof\textsuperscript{a}. Dr\textsuperscript{a}. Marilisa Shimazumi

    \vspace{3cm}
    \makebox[0.75\textwidth]{\hrulefill}\\
    %Prof. Simone Sarmento, Ph.D.
    Prof\textsuperscript{a}. Dr\textsuperscript{a}. Simone Sarmento

\end{titlepage}

\begin{titlepage}
    \begin{center}
        \vspace*{11cm}
        %{To Letícia}
        {A Letícia}
        \vfill
    \end{center}
\end{titlepage}

\begin{titlepage}
    \vspace*{10cm}
    \begin{adjustwidth}{0.5\textwidth}{0cm}
        %The present work was carried out with the support of the National Council for Scientific and Technological Development (CNPq), Grant \#131464/2023-0.
        O presente trabalho foi realizado com o apoio do Conselho Nacional de Desenvolvimento Científico e Tecnológico (CNPq), Processo 131464/2023-0.
    \end{adjustwidth}
\end{titlepage}

\begin{titlepage}
    \begin{center}
        %{\large \MakeUppercase{Acknowledgements}}
        {\large \MakeUppercase{Agradecimentos}}
    \end{center}
    \vspace{1cm}
    %I would like to express my deepest gratitude to all the professors in the Applied Linguistics and Language Studies Graduate Programme (LAEL) for serving as my mentors in this field.
    Gostaria de expressar minha mais profunda gratidão a todos os professores do Programa de Estudos Pós-Graduados em Linguística Aplicada e Estudos da Linguagem (LAEL) por atuarem como meus mentores nesta área.

    %A special acknowledgement goes to Professor Tony Berber Sardinha, my supervisor for this research project, who has always been an inspiring role model to me.
    Um agradecimento especial ao Professor Tony Berber Sardinha, meu orientador neste projeto de pesquisa, que sempre foi para mim um modelo inspirador.

    %I also wish to thank Professor Maria Cláudia Nunes Delfino and Professor Carlos Henrique Kauffmann for their encouragement.
    Também desejo agradecer à Professora Maria Cláudia Nunes Delfino e ao Professor Carlos Henrique Kauffmann pelo incentivo.

    %Lastly, I would like to extend my heartfelt thanks to all my colleagues and staff members who accompanied and supported me throughout my journey.
    Por fim, gostaria de expressar meus sinceros agradecimentos a todos os colegas e membros da equipe que me acompanharam e apoiaram ao longo da minha trajetória.
\end{titlepage}

\begin{titlepage}
    \vspace*{\fill}
    %\epigraph{`[R]egisters' are cultural categories, not scientific categories. These categories can be described for their typical situational and linguistic characteristics. But they are not defined in those terms. In fact, registers do not have definitions in terms of their necessary characteristics. Rather, cultures and languages evolve naturally in terms of such categorical organizations, without any scientific basis.}{\citet[p.~16]{biberWhatRegisterAccounting2023}}
    \epigraph{`[R]egistros' são categorias culturais, não categorias científicas. Essas categorias podem ser descritas por suas características situacionais e linguísticas típicas. Mas elas não são definidas nesses termos. Na verdade, registros não possuem definições em termos de características necessárias. Em vez disso, culturas e línguas evoluem naturalmente em termos de tais organizações categóricas, sem qualquer fundamento científico.}{\citet[p.~16]{biberWhatRegisterAccounting2023}}
\end{titlepage}

\selectlanguage{brazilian}
\title{\Large \MakeUppercase{Ferramentas baseadas em IA como Assistentes de Escrita Acadêmica em Língua Inglesa para Autores em Inglês como Segunda Língua: Uma Análise Multidimensional}}
\author{Rogério Yamada} % Your name here
\date{2025}
\maketitle
\renewcommand{\abstractname}{Resumo}
\begin{abstract}
    Pesquisadores que não têm o inglês como língua materna frequentemente precisam escrever em inglês, pois a predominância do inglês na academia o estabeleceu como o principal meio de publicação em periódicos acadêmicos de qualidade \citep{belcherSeekingAcceptanceEnglishonly2007, baumvolScholarlyPublicationBrazilian2021, flowerdewEnglishResearchPublication2012, cargillIntroductionSpecialIssue2008}. No entanto, o desafio reside em utilizar uma linguagem idiomática que atenda às expectativas de registro do inglês acadêmico. O registro acadêmico em inglês baseia-se em padrões retóricos e lexicogramaticais específicos \citep{biberLexicalBundlesUniversity2007,adelRecurrentWordCombinations2012}, que diferem de muitos dos presentes na formação acadêmica dos autores \citep{adelRecurrentWordCombinations2012, pangLexicalBundlesConstruction2010, wrayConcludingQuestionWhy2019}. Partindo do pressuposto de que Modelos Amplos de Linguagem, como o ChatGPT, foram extensivamente treinados no inglês acadêmico, este estudo buscou verificar até que ponto ferramentas baseadas em IA são capazes de ajudar escritores em inglês como L2 a atender às necessidades lexicogramaticais e retóricas do inglês acadêmico. Pesquisas anteriores sobre o inglês acadêmico gerado por IA sugerem que tais ferramentas não necessariamente reproduzem as escolhas retóricas ou lexicogramaticais dos autores humanos \citep{berbersardinhaAIgeneratedHumanauthoredTexts2024}. Para aprofundar nessa questão, o estudo empregou um corpus de textos produzidos por falantes de inglês como L2 (English-as-L2 Authored Papers -- EL2AP), compilado a partir do repositório SciELO Pré-prints \citep{SciELOPreprints}, e um corpus de textos publicados em periódicos de qualidade (Quality Journals Published Papers -- QJPP), refletindo as mesmas disciplinas do EL2AP. Foram incluídos apenas artigos submetidos antes do advento do ChatGPT. Os textos do EL2AP foram revisados com o ChatGPT e reunidos no corpus AI-EL2AP. Para avaliar as similaridades e diferenças entre textos produzidos por humanos e por IA, foi realizada uma Análise Multidimensional \citep{biberVariationSpeechWriting1988, biberDimensionsRegisterVariation1995} por meio de uma análise aditiva \citep{berbersardinhaAddingRegistersPrevious2019}. De modo geral, os resultados indicaram que a redação acadêmica assistida por IA diverge dos padrões humanos ao recorrer a padrões não típicos do inglês acadêmico.

    \vspace{1em}
    \textbf{Palavras-chave}: Análise Multidimensional; Autores em inglês como segunda língua; Linguagem gerada por IA; Registro inglês acadêmico
\end{abstract}

\selectlanguage{english}
\title{\Large \MakeUppercase{AI-Powered Tools as Academic English Writing Assistants to English-as-L2 Authors: A Multi-Dimensional Analysis}}
\author{Rogério Yamada} % Your name here
\date{2025}
\maketitle
\begin{abstract}
    Non-English-speaking scholars are often required to write in English, as the predominance of English in academia has established it as the primary medium for quality academic journals \citep{belcherSeekingAcceptanceEnglishonly2007, baumvolScholarlyPublicationBrazilian2021, flowerdewEnglishResearchPublication2012, cargillIntroductionSpecialIssue2008}. However, the challenge lies in writing idiomatic language that meets the register expectations of academic English. The academic register in English relies on specific rhetorical and lexicogrammatical patterns \citep{biberLexicalBundlesUniversity2007,adelRecurrentWordCombinations2012} that differ from many of those in their authors’ academic background \citep{adelRecurrentWordCombinations2012, pangLexicalBundlesConstruction2010, wrayConcludingQuestionWhy2019}. Assuming that Large Language Models such as ChatGPT have been extensively trained on academic English, this study aimed to verify the degree to which AI-powered tools are capable of assisting English-as-L2 writers in meeting the lexicogrammatical and rhetorical needs of academic English. Previous research in AI-generated academic English suggests that AI tools do not necessarily reproduce the rhetorical or lexicogrammatical choices of human authors \citep{berbersardinhaAIgeneratedHumanauthoredTexts2024}. To further explore this, the study employed an English-as-L2-Authored Papers (EL2AP) corpus compiled from the SciELO Preprints archive \citep{SciELOPreprints} and a Quality Journals Published Papers (QJPP) corpus reflecting the same disciplines as EL2AP. Only articles submitted before the advent of ChatGPT were included. EL2AP texts were revised with ChatGPT and compiled into the AI-EL2AP corpus. To gauge the similarities and differences between human- and AI-generated texts, a Multi-Dimensional Analysis \citep{biberVariationSpeechWriting1988, biberDimensionsRegisterVariation1995} was carried out through an additive analysis \citep{berbersardinhaAddingRegistersPrevious2019}. Overall, the results indicated that AI-assisted academic writing diverges from human standards by relying on non-typical patterns of academic English.

    \vspace{1em}
    \textbf{Keywords}: AI-generated language; English academic register; English-as-L2 authors; Multi-Dimensional Analysis
\end{abstract}

\selectlanguage{brazilian} % Comment this line if the document is expected to be in English

\listoffigures
\thispagestyle{empty}
\listoftables
\thispagestyle{empty}
\lstlistoflistings
\thispagestyle{empty}
\tableofcontents
\thispagestyle{empty}
\clearpage
\pagenumbering{arabic}
\setcounter{page}{16} % Set the initial page number
