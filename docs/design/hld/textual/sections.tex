\section{Project Overview}
\subsection{Purpose}
This document describes the high-level design of a Python-based application for Lexical Multi-Dimensional Analysis (LMDA) (Figure \ref{fig:lexical_md_analysis}) with an interactive graphical user interface (GUI). The system ingests text corpora, performs linguistic preprocessing, extracts lexical features, projects texts into a multi-dimensional factor space via Factor Analysis (FA), and presents results through interactive visualisations and reports (Figures \ref{fig:lexical_md_analysis_procedures} and \ref{fig:from_text_to_factor}).

\import{diagrams/}{md_analysis_diagram_1_lexical}

\import{diagrams/}{md_analysis_diagram_lexical_procedures}

\import{diagrams/}{from_text_to_factor}

\subsection{Scope}
The application supports:
\begin{itemize}
    \item Corpus ingestion from files, folders, and CSV/TSV and JSONL columns;
    \item Language-aware preprocessing (tokenisation, lemmatisation, POS tagging);
    \item Feature extraction (lexical richness, frequency profiles, function/content word ratios, POS n-grams, domain lexicons);
    \item Dimensionality reduction via Factor Analysis to derive interpretable dimensions;
    \item Interactive exploration (scatter plots, loadings, factor scores), filtering, and labelling;
    \item Exportable reports (PDF/HTML) and machine-readable outputs (CSV/TSV/JSON/JSONL).
\end{itemize}

\subsection{Stakeholders}
\begin{itemize}
    \item Linguists and corpus analysts;
    \item Data scientists working on stylistic and register analysis;
    \item Educators and applied researchers;
    \item Software maintainers and operations personnel (if deployed as a desktop or server app).
\end{itemize}

\section{Functional Requirements}
\subsection{Corpus Ingestion}
\begin{itemize}
    \item FR-1: Import plain text files, folders, and CSV/TSV and JSONL with a designated text column;
    \item FR-2: Detect/allow selection of text encoding (UTF-8 default), handle BOM (Byte Order Mark), and report decoding errors;
    \item FR-3: Optional metadata import (e.g., author, source, label) to support grouping and filtering.
\end{itemize}

\subsection{Preprocessing}
\begin{itemize}
    \item FR-4: Tokenise, normalise (case, punctuation policies), and sentence-segment texts;
    \item FR-5: Lemmatise and POS-tag (TreeTagger; spaCy models; fallback to NLTK if unavailable);
    \item FR-6: Language selection and multi-language support; warn if models are missing;
    \item FR-7: Stopword management (built-in lists and user-defined additions).
\end{itemize}

\subsection{Feature Extraction}
\begin{itemize}
    \item FR-8: Use only content-word lemmas (based on POS) for candidate variables;
    \item FR-9: Select exactly K lemmas (configurable; default K=1,000) as variables for downstream analysis;
    \item FR-10: Support two selection strategies:
    \begin{itemize}
        \item Top-K frequent lemmas corpus-wide (with deterministic tie-breaking);
        \item Top-K keywords (Figure \ref{fig:lexical_variables_selection_by_keywords}) via cross-group comparisons (e.g. by source/label) using configurable statistics (log-likelihood, chi-square, log-odds with smoothing), with expected-frequency filters and deterministic tie-breaking;
    \end{itemize}
    \item FR-11: Allow optional thresholds for candidate inclusion (minimum document frequency, minimum total frequency) and language-aware content-word definitions;
    \item FR-12: For the selected K lemmas, compute per-document counts and normalise to per-thousand tokens (retain raw counts; optionally provide z-scored variants)
    \item FR-13: Persist the selected vocabulary with ranks/scores and enable ``frozen vocabulary'' reuse across runs for compatibility;
\end{itemize}

\import{diagrams/}{lexical_variables_selection_by_keywords}

\import{diagrams/}{lexical_variables_selection_by_correlation}

\subsection{Dimensional Analysis}
\begin{itemize}
    \item FR-14: Build a document-feature matrix (DFM) with configurable weighting (raw, tf-idf, z-scores);
    \item FR-15: Perform dimensionality reduction with Exploratory Factor Analysis (EFA/Varimax);
    \item FR-16: Compute factor loadings, factor scores, and explained variance;
    \item FR-17: Save and load trained models/pipelines for reproducibility.
\end{itemize}

\subsection{Visualisation and GUI}
\begin{itemize}
    \item FR-18: Interactive scatter/biplots of documents in 2D/3D factor space with tooltips and labels;
    \item FR-19: Loading plots, scree plots, and feature contribution charts;
    \item FR-20: Brushing/filtering by metadata, search, and selection for detailed inspection;
    \item FR-21: Theming, layout persistence, and session autosave.
\end{itemize}

\subsection{Export and Reporting}
\begin{itemize}
    \item FR-22: Export factor scores, loadings, and projections to CSV/JSON;
    \item FR-23: Report the N (configurable; default N=50) top scoring texts in each factor pole;
    \item FR-24: Generate PDF/HTML reports with key plots and tables;
    \item FR-25: Export model artefacts (preprocessing config, feature schema, EFA components).
\end{itemize}

\subsection{Automation and Scripting}
\begin{itemize}
    \item FR-26: Batch processing via CLI and/or headless mode;
    \item FR-27: Optional REST API for programmatic access (server mode).
\end{itemize}

\section{Non-Functional Requirements}
\subsection{Performance}
\begin{itemize}
    \item NFR-1: Handle corpora up to 1M tokens on a standard laptop with reasonable latency (\textless 10s for EFA on 10k documents x 1k features, assuming sparse ops);
    \item NFR-2: Incremental processing and caching to avoid recomputation.
\end{itemize}

\subsection{Usability}
\begin{itemize}
    \item NFR-3: Intuitive GUI with sensible defaults; accessible color schemes;
    \item NFR-4: Contextual help and onboarding tips; undo/redo for key actions.
\end{itemize}

\subsection{Reliability and Robustness}
\begin{itemize}
    \item NFR-5: Deterministic pipelines when seeds are fixed; versioned model artifacts;
    \item NFR-6: Error handling with clear diagnostics and recovery suggestions.
\end{itemize}

\subsection{Portability and Compatibility}
\begin{itemize}
    \item NFR-7: Cross-platform support (Windows, macOS, Linux);
    \item NFR-8: Python 3.10+; standalone (PyInstaller); optional packaged with virtualenv/Poetry.
\end{itemize}

\subsection{Security}
\begin{itemize}
    \item NFR-9: Local-only data processing by default; no data leaves the machine;
    \item NFR-10: If server mode is enabled, enforce TLS and authentication.
\end{itemize}

\section{System Architecture}
\subsection{Overview}
The system follows a modular, pipeline-oriented architecture separating ingestion, preprocessing, feature extraction, modelling, and presentation layers. A central configuration orchestrates reproducible runs; artefacts are persisted for reuse.

\subsection{Components}
\paragraph{GUI Layer}
Desktop GUI (PySide6/PyQt) for user workflows: project management, corpus loading, configuration, visualisation, and export.

\paragraph{Controller / Orchestration}
Coordinates pipeline stages, manages state, triggers incremental recomputation when inputs change, and logs provenance.

\paragraph{Preprocessing Service}
Language detection, sentence segmentation, tokenisation, lemmatisation, POS tagging, stopword filtering; caches normalised representations.

\paragraph{Feature Service}
Computes metrics and vectorises documents into a DFM with configurable normalisation and weighting; supports plug-ins.

\paragraph{Modeling Service}
Performs EFA, computes loadings and scores, applies rotations (e.g., Varimax/Promax), and persists models.

\paragraph{Visualisation Service}
Generates interactive plots (2D/3D scatter, biplots, scree, loadings) and data tables; supports selections and linked views.

\paragraph{Persistence Layer}
Stores corpora metadata, preprocessed artefacts, features, and models (e.g., SQLite for metadata, parquet/npz for matrices, joblib for models).

\paragraph{API Layer (Optional)}
FastAPI-based REST endpoints for batch/remote processing in server mode.

\subsection{Data Flow}
\begin{enumerate}
    \item Import corpus $\rightarrow$ normalise text $\rightarrow$ linguistic annotation.
    \item Extract features $\rightarrow$ construct DFM $\rightarrow$ apply weighting/normalisation.
    \item Fit EFA $\rightarrow$ compute scores/loadings $\rightarrow$ store artifacts.
    \item Visualise and interact $\rightarrow$ export results and reports.
\end{enumerate}

\section{Technology Stack}
\begin{itemize}
    \item Language: Python 3.12 (or 3.10+);
    \item GUI: PySide6 or PyQt6; alternative: Qt for Python widgets + matplotlib/seaborn/plotly;
    \item NLP: TreeTagger (via treetaggerwrapper), spaCy (core models: \texttt{en\_core\_web\_sm/md/lg}), NLTK (fallback, resources), langdetect/fastText for language ID;
    \item Data: pandas, numpy, scipy (sparse matrices), factor analysers (\texttt{factor\_analyzer}) for EFA/rotations;
    \item Visualisation: matplotlib, seaborn, plotly; pyqtgraph for fast interactive plots;
    \item Persistence: SQLite (metadata), parquet/CSV (tables), npz (sparse matrices), joblib/pickle (models), YAML/TOML (config);
    \item Packaging: Poetry or pip-tools; PyInstaller for desktop bundle; Docker for server mode;
    \item Testing: pytest, hypothesis for property-based testing, tox/\texttt{nox} for matrix runs;
    \item CI/CD: GitHub Actions or GitLab CI for tests, linting (ruff, black), and packaging.
\end{itemize}

\section{Interfaces and APIs}
\subsection{Python Module Interfaces}
\begin{itemize}
    \item \textbf{Preprocessor}: \texttt{process(corpus, config) $\rightarrow$ annotations};
    \item \textbf{FeatureExtractor}: \texttt{fit\_transform(annotations) $\rightarrow$ DFM}; \texttt{transform(...)};
    \item \textbf{Modeler}: \texttt{fit(DFM) $\rightarrow$ model}; \texttt{transform(DFM) $\rightarrow$ scores};
    \item \textbf{Visualiser}: \texttt{plots(scores, loadings, metadata)};
    \item \textbf{Persistence}: \texttt{save\_artifact(obj, kind)} and \texttt{load\_artifact(kind)}.
\end{itemize}

\subsection{Plug-in Interface}
Third-party feature extractors implement \texttt{IFeaturePlugin}:
\begin{itemize}
    \item \texttt{schema()} returns feature names and dtypes;
    \item \texttt{compute(doc)} yields a sparse/compact feature vector;
    \item Registration via entry points or a plugins folder.
\end{itemize}

\subsection{CLI}
\begin{verbatim}
lmda run --config config.yaml --input corpus/ --output artifacts/
lmda visualise --project my_project.lmda
\end{verbatim}

\subsection{Optional REST API (FastAPI)}
\begin{itemize}
    \item POST /v1/projects: create project
    \item POST /v1/projects/{id}/ingest: upload texts/metadata
    \item POST /v1/projects/{id}/run: execute pipeline
    \item GET /v1/projects/{id}/scores: factor scores
    \item GET /v1/projects/{id}/loadings: factor loadings
    \item GET /v1/projects/{id}/report: PDF/HTML report
\end{itemize}

\section{Security and Privacy Considerations}
\subsection{Data Protection}
\begin{itemize}
    \item Local-first processing; no external calls by default;
    \item Optional at-rest encryption for project files; secure temp directories;
    \item Configurable redaction of personally identifiable information (PII) in exports.
\end{itemize}

\subsection{Access Control (Server Mode)}
\begin{itemize}
    \item Authentication (OAuth2/JWT) and role-based access control (RBAC);
    \item TLS enforcement; secure headers and CSRF protections for Web GUI (if used).
\end{itemize}

\subsection{Supply Chain}
\begin{itemize}
    \item Pin dependencies; verify wheels; use vulnerability scanning (pip-audit, Safety);
    \item Sandboxed plug-ins with permission checks.
\end{itemize}

\section{Risks and Mitigations}
\begin{itemize}
    \item \textbf{R1: Model interpretability} — Factors may be hard to interpret. \\
          \emph{Mitigation}: Offer rotations, feature contribution charts, and glossary tooltips.
    \item \textbf{R2: Performance on large corpora} — High memory/CPU usage. \\
          \emph{Mitigation}: Sparse structures, incremental processing, caching, chunked I/O.
    \item \textbf{R3: NLP model availability} — TreeTagger and spaCy models not installed. \\
          \emph{Mitigation}: Guided download, fallback tokenisers, graceful degradation.
    \item \textbf{R4: Data privacy} — Sensitive texts accidentally exported. \\
          \emph{Mitigation}: Local-first defaults, explicit consent prompts, export redaction.
    \item \textbf{R5: Cross-platform GUI quirks} — Rendering differences. \\
          \emph{Mitigation}: UI testing matrix, use of Qt styles, theming validation.
    \item \textbf{R6: Reproducibility drift} — Version changes affect outputs. \\
          \emph{Mitigation}: Lockfiles, artefact versioning, seeds, environment capture.
\end{itemize}

\section{Timeline and Milestones}
\begin{enumerate}
    \item \textbf{Week 1--2: Foundations} \\
          Project scaffolding, env setup, CI/CD, basic data model, configuration schema;
    \item \textbf{Week 3--4: Preprocessing} \\
          Language detection, tokenisation, lemmatisation, POS tagging; caching;
    \item \textbf{Week 5--6: Features} \\
          Core lexical metrics, function word profiles, POS n-grams; plug-in API;
    \item \textbf{Week 7--8: Modeling} \\
          EFA pipeline, rotations, persistence; basic CLI;
    \item \textbf{Week 9--10: GUI v1} \\
          Data loading, configuration panels, 2D scatter, scree and loadings plots;
    \item \textbf{Week 11--12: Reporting \& Export} \\
          CSV/JSON exports, PDF/HTML reports; theming and preferences;
    \item \textbf{Week 13--14: Hardening} \\
          Performance tuning, testing, error handling, and documentation;
    \item \textbf{Week 15+: Optional Server Mode} \\
          FastAPI endpoints, auth/TLS, Docker packaging.
\end{enumerate}
