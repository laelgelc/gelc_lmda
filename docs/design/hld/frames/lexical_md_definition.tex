\begin{frame}{Lexical MD Analysis}
    \scriptsize
    \vspace*{-4em}

    \begin{exampleblock}{\cite[p.1]{berbersardinhaLexicalMultiDimensionalAnalysis2025}}
    Lexical multidimensional analysis (LMDA), an extension of the multidimensional (MD) analysis framework developed by Biber in the 1980s (``multi-feature multidimensional analysis'') to study register variation. Through the identification of (lexical) dimensions or sets of correlated lexical features, LMDA enables the analysis of lexical patterning from a multidimensional perspective. These lexical dimensions represent a variety of latent, macro-level discursive constructs.
    \end{exampleblock}

    \begin{itemize}
    \item Lexis as entry-point to discourse: Lexis is a suitable level of analysis for identifying large-scale ideological discourses (p.9)
    \item Lexical priming and text: Lexical priming \citep{hoeyLexicalPrimingNew2007} explains collocation as psychologically conditioned by prior exposure and sensitive to register and discourse function (p.8)
    \item Lexical patterns reveal ideological discourse: Lexical co-occurrence patterns in texts can reveal ideological discourses when analyzed through a corpus-based lens (p.9)
    \item Lexical patterns are register sensitive: Because lexical associations vary by register, social and contextual meaning is embedded in lexical distributions (p.9)
    \item Discourse patterns emerge from lexis-register patterns: Discourses emerge from patterns of lexical co-occurrence in association with register (p.9)
    \end{itemize}
\end{frame}





